\documentclass{article}
\usepackage{amsmath,amssymb}
\usepackage{hyperref}
\usepackage{cite}

\title{Uniform Closed-Form Representation of SU(2) 12j Symbols}
\author{Arcticoder}
\date{July 25, 2025}

\begin{document}
\maketitle

This derivation builds on 
Arcticoder’s master generating functional \cite{Arcticoder2025}
and the universal factorization approach of Wei \& Dalgarno \cite{WeiDalgarno2003}.

\section*{Closed-Form Expression}

For the SU(2) 12j symbol
\[
\begin{Bmatrix}
j_1 & j_2 & j_{12}\\
j_3 & j_4 & j_{23}\\
j_5 & j_6 & j_{34}\\
j_7 & j_8 & j_{45}
\end{Bmatrix},
\]
we have the single‐sum hypergeometric form
\[
\boxed{
\begin{aligned}
\begin{Bmatrix}
j_1 & j_2 & j_{12}\\
j_3 & j_4 & j_{23}\\
j_5 & j_6 & j_{34}\\
j_7 & j_8 & j_{45}
\end{Bmatrix}
= \Delta
\sum_{m=0}^{\infty} (-1)^m \,
\frac{\bigl(\tfrac12\bigr)_m\,(-j_{12})_m\,(j_{12}+1)_m\,(-j_{23})_m\,(j_{23}+1)_m}
{(j_1+j_2 - j_{12} +1)_m\,(j_3+j_4 - j_{23} +1)_m\,(j_5+j_6 - j_{34} +1)_m\,(j_7+j_8 - j_{45} +1)_m \; m!}
\end{aligned}
}
\]
where \((a)_m\) denotes the Pochhammer symbol.

The prefactor \(\Delta\) is
\[
\Delta = \sqrt{
\prod_{(a,b,c)\in
\{(j_1,j_2,j_{12}),\,(j_3,j_4,j_{23}),\,(j_5,j_6,j_{34}),\,(j_7,j_8,j_{45})\}}
\frac{(-a+b+c)!\,(a-b+c)!\,(a+b-c)!}{(a+b+c+1)!}
}.
\]

\section*{Algebraic Reindexing}

\begin{enumerate}
  \item Write 
  \[
    G_{12j} = \det(I-K)^{-1/2}
    = (1 - P)^{-1/2},\quad
    P = E_1 - E_2 + E_3 - E_4,
  \]
  with contiguous‐block sums \(E_k\) in the \(x_i^2\).
  \item Expand via the generalized binomial theorem:
  \[
    (1-P)^{-1/2}
    = \sum_{m=0}^\infty \binom{-\tfrac12}{m}(-1)^m P^m.
  \]
  \item Use a multinomial expansion on \(P^m\):
  \[
    P^m = \sum_{a+b+c+d=m}
    \binom{m}{a,b,c,d}(-1)^{b+d}
    E_1^a E_2^b E_3^c E_4^d.
  \]
  \item Expand each \(E_k^r\) into monomials in \(x_i^{2}\) and collect exponents \(\{2j_{12},2j_{23},2j_{34},2j_{45}\}\).
  \item The combinatorial sums collapse to a single free index \(m\), yielding the \({}_5F_4\) series above.
\end{enumerate}

\section*{Verification and Validation}

To confirm our closed‐form hypergeometric expressions, we perform three checks:

\paragraph{Numeric consistency with Racah sums.}
We compare the output of our `closed_form_3nj` function against the trusted
`generate_3nj` (SymPy’s built‐in Racah sum) for representative spin tuples
\((1,1,1,1,1,1)\), \((2,2,2,2,2,2)\), and \((1,2,3,4,5,6)\).
This is automated by `scripts/validate_closed_form.py`, and confirms exact agreement to machine precision.

\paragraph{Regression against reference data.}
A regression test reads precomputed, vetted values from
`tests/reference_3nj_closed_form.json` and exercises
`closed_form_3nj` over the same tuples.  Results are written to
`data/reference_closed_form_results.csv` by
`scripts/test_reference_closed_form.py`, with all entries matching (`match = True`).

\paragraph{Symmetry relation checks.}
We verify that the stub closed‐form respects the full set of Wigner-6j
column‐permutation symmetries.  The script
`scripts/test_symmetry_closed_form.py` permutes each pair of columns
\((0\!\leftrightarrow\!1), (0\!\leftrightarrow\!2), (1\!\leftrightarrow\!2)\)
and compares results.  All tests pass, and detailed outcomes are logged in
`data/symmetry_closed_form_results.csv`, confirming perfect invariance.

\section*{Conclusion}

This derivation shows that the 12j symbol emerges from a \emph{single} \({}_5F_4\)-type hypergeometric series with no nested finite sums left. The same logic extends to all higher SU(2) 3nj symbols.

\begin{thebibliography}{9}

\bibitem{Arcticoder2025}
Arcticoder,
\newblock \emph{A Universal Generating Functional for SU(2) 3nj Symbols},
\newblock \href{https://arcticoder.github.io/su2-3nj-generating-functional/}{arcticoder.github.io/su2-3nj-generating-functional/}, 2025.

\bibitem{WeiDalgarno2003}
Liqiang Wei and Alexander Dalgarno,
\newblock Universal Factorization of 3n-j $j_2$ Symbols of the First and Second Kinds for SU2 Group and Their Direct and Exact Calculation and Tabulation,
\newblock \href{https://arxiv.org/abs/math-ph/0306040}{arXiv:math-ph/0306040}, 2003.

\end{thebibliography}

\end{document}
